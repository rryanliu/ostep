\documentclass{article}
\usepackage{amsmath, amssymb, graphicx, parskip}
\title{Interlide: Process API}
\author{Ryan Liu}

\newtheorem{theorem}{Theorem}[section]
\newtheorem{claim}{Claim}[section]
\newtheorem{subclaim}{Subclaim}[claim]
\newtheorem{corollary}{Corollary}[theorem]
\newtheorem{lemma}[theorem]{Lemma}

\begin{document}
\maketitle
\thispagestyle{empty}
\newpage
\pagenumbering{arabic}

\setcounter{section}{4}
\section*{}

To create a process in UNIX, use a pair of system calls: \textsf{fork()} and \textsf{exec()}.
A third call, \textsf{wait()} can be used by a process wishing to wait for a process it has created to complete.

\subsection{The \textsf{fork()} System Call}

\textsf{fork()} is lwk weird.
We learned from Tanenbaum that \textsf{fork()} creates a clone of the current process.
But furthermore, the process that is created doesn't start running at \textsf{main()}, but it comes into life as if it had called \textsf{fork()} itself --- a true clone, a copy of the array of bytes representing the entirety of the state of the parent process.

But it still isn't an \textit{exact} copy, like how a clone is now another individual.
The value that \textsf{fork()} returns to the parent and child is different.
While the parent receives the PID of the child, the child is returned a 0.
This is useful because now we know which is parent and which is child.

\subsection{Adding \textsf{wait()} System Call}

This is actually not deterministic and can lead to race conditions, unless we use \textsf{wait()} of \textsf{waitpid()}

\subsection{Finally, the \textsf{exec()} System Call}

There are 6 variants of \textsf{exec()}: \textsf{exec1(), execle(), execlp(), execv(), execvp()}.
Read the man pages

\end{document}
